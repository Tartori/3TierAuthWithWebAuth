\usepackage{glossaries}


\newglossaryentry{fido2}{name=FIDO2, description=New standard from the FIDO Alliance which combines CTAP and WebAuthn}
\newacronym{fido}{FIDO}{Fast IDentity Online}
\newacronym{ctap}{CTAP}{Client-to-Authenticator Protocol}
\newacronym{nist}{NIST}{National Institute of Standards and Technology}
\newacronym{webAuthn}{WebAuthn}{Web Authentication}
\newacronym{u2f}{U2F}{Universal 2nd Factor}
\newacronym{rest}{REST}{REpresentional State Transfer}
\newacronym{tls}{TLS}{Transport Layer Security}
\newacronym{api}{API}{Application Programming Interface}
\newacronym{json}{JSON}{JavaScript Object Notation}
\newacronym{cose}{COSE}{CBOR Object Signing and Encryption (COSE)}
\newacronym{cbor}{CBOR}{Concise Binary Object Representation}
\newglossaryentry{sha-256}{name=SHA-256, description={Secure Hash Algorithm 2 with 256 Bit Output}}
\newacronym{utf-8}{UTF-8}{Universal Transformation Format}
\newacronym{nfc}{NFC}{Near Field Communication}
\newacronym{usb}{USB}{Universal Serial Bus}
\newacronym{eid}{eID}{electronical IDentitification}
\newacronym{hlos}{HLOS}{High Level Operating System}
\newacronym{aroe}{AROE}{Allowed Restricted Operating Environment}
\newacronym{mitm}{MitM}{Man in the Middle}
\newacronym{loa}{LoA}{Level of Assurance}
\newacronym{rp}{RP}{Relying Party}
\newglossaryentry{base64}{
name={Base 64},
  description={Base 64 is a binary-to-text encoding that uses 64 ASCII characters to represent 6 bit. This results in an overhead of 2 bit per byte. There is also a Base 64 URL encoding which uses slightly different characters.}}
\newglossaryentry{csa}{name={credential stuffing attacks}, description={Using credentials that were leaked on one site to access other sites. This is often an attack against passwords when passwords are reused}}

